\documentclass{article}

\usepackage[utf8]{inputenc}
\usepackage{amsmath}
\usepackage{graphicx}
\usepackage[hidelinks]{hyperref}
\usepackage{hyperref}
\usepackage[backend=biber, style=numeric]{biblatex}
\addbibresource{refs.bib}
\usepackage[]{geometry}

\newgeometry{vmargin={15mm}, hmargin={30mm}}

\title{Portfolio Specification}
\author{Jack Foley | C00274246 \\[1cm]Supervisor: Greg Doyle}
\date{\today}

\begin{document}

\maketitle

\newpage

\tableofcontents

\newpage

\section{Introduction}
This is a technical specification document for the Data Science (DS) and Machine Learning (ML) 
module run by Greg Doyle. As a part of the module, we, the students, are required to 
create this document to plan the various machine learning and data science projects 
that we will be working on throughout the first semester. By the end of the semester
we are expected to have completed all the projects defined in this document and then
present them to the class. This document will be used as a reference throughout the
semester to ensure that we are on track to complete the projects on time. 

\section{Ideas}

\subsection{Sea Level Prediction}

\begin{itemize}
    \item \textbf{Brief Description:} Predict future sea levels using a polynomial regression algorithm.
    \item \textbf{Dataset Source:} \href{https://www.kaggle.com/datasets/kkhandekar/global-sea-level-1993-2021}{https://www.kaggle.com/datasets/kkhandekar/global-sea-level-1993-2021} 
    \item \textbf{Technologies:}
    \begin{itemize}
        \item Python
        \item Pandas
        \item Numpy
        \item Matplotlib
        \item Scikit-learntitanic(1)
        \item Jupyter Notebook
        \item Polynomial Regression
    \end{itemize}
    \item \textbf{Detailed Description:} In a world where global warming is steadily heating up the planet, a side affect is the rise of the sea levels across the globe. 
    It is important that we have a prediction of where the sea levels will end up in the future so that we can prepare for the worst. To do this, we can use a polynomial 
    regression algorithm to predict future sea levels. The dataset will have to be cleaned and preprocessed before we can use it to train the model.
\end{itemize}

\subsection{Titanic Death Prediction}

\begin{itemize}
    \item \textbf{Brief Description:} Predict whether a passenger on the Titanic survived or not.
    \item \textbf{Dataset Source:} \href{https://web.stanford.edu/class/archive/cs/cs109/cs109.1166/problem12.html}{https://web.stanford.edu/class/archive/cs/cs109/cs109.1166/problem12.html}
    \item \textbf{Technologies:}
    \begin{itemize}
        \item Python
        \item Pandas
        \item Numpy
        \item Matplotlib
        \item Scikit-learn
        \item Jupyter Notebook
        \item Possible ML Models:
        \begin{itemize}
            \item Decision Tree
        \end{itemize}
    \end{itemize}
\end{itemize}

\newpage
\printbibliography

\end{document}